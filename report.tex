%------------------------------------------------------------
\documentclass[a4paper,draft, 11pt]{article}
%----------------------------------------------------------

\usepackage[utf8x]{inputenc}
\usepackage{amsmath,amssymb,amsthm,amsfonts}
\usepackage[left=3.5cm, right=2cm]{geometry}
\usepackage{fullpage}
\usepackage{amsmath}
%\usepackage{subfigure}
\usepackage[pdftex]{graphicx}%image support.
\usepackage{comment}
\usepackage{natbib}
\usepackage{fancyhdr}
%\usepackage[hidelinks]{hyperref}
\usepackage{setspace}

\onehalfspacing

\setlength{\headsep}{15pt}
\setlength{\headheight}{15pt}
\pagestyle{fancy}
\lhead{}\chead{}\rhead{17746}
\lfoot{}\cfoot{\thepage}\rfoot{}

\begin{document}

\title{Department of Statistics 2014\\ {\bf Project in Computational Finance and Insurance}\\ {\small Submitted for the Master of Science, London School of Economics} }
\author{Candidate Number: \bf 17746}
%\date{}
\maketitle
\vfill
\section{Abstract}
This project will simlulate several interesting models used in financial markets. We will start with the famous Black-Scholes' model with constant volatility. Secondly, Heston model will introduce stochastic volatility into the Black-Scholes' system. We will discuss some application of these models with respect to option pricing.
We will also discuss a simple model in ruin theory and systematic risk in banking system from stochastic approach.

\newpage

\section{Introduction}
% why simulation, benefit, ??
The report will focus mainly on three topics 

\section{Black-Scholes' Economy}
The most basic model in the stochastic approach to model stock price is the Black-Schole-Merton Model. In Black-Scholes' economy, the stock price $(S_t)$ is given as
$$S_t = S_0 exp\left( (r - \frac{\sigma^2}{2} )t + \sigma W_t \right)$$
where $r$ is the interest rate, sigma is the volatility of that stock and $W$ is the standard Brownian motion.



\section{conclusion}




%\bibliographystyle{agsm}
%\bibliography{Reference}

\end{document}

