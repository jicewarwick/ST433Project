\documentclass[a4paper, 11pt]{article}

\usepackage[utf8x]{inputenc}
\usepackage{amsmath,amssymb,amsthm,amsfonts}
\usepackage[left=3.5cm, right=2cm]{geometry}
\usepackage{fullpage}
\usepackage{amsmath}
%\usepackage{subfig}
\usepackage[pdftex]{graphicx}%image support.
\usepackage{caption}
\usepackage{subcaption}
\usepackage{comment}
\usepackage{natbib}
\usepackage{fancyhdr}
\usepackage[hidelinks]{hyperref}
\usepackage{setspace}

\onehalfspacing

\setlength{\headsep}{15pt}
\setlength{\headheight}{15pt}
\pagestyle{fancy}
\lhead{}\chead{}\rhead{17746}
\lfoot{}\cfoot{\thepage}\rfoot{}

\graphicspath{{./fig}}

\begin{document}

\title{Department of Statistics 2014\\ {\bf Project in Computational Finance and Insurance}\\ {\small Submitted for the Master of Science, London School of Economics} }
\author{Candidate Number: \bf 17746}
%\date{}
\maketitle
\vfill
\section*{Abstract}
This project will simlulate several interesting models used in financial markets. We will start with the famous Black-Scholes' model with constant volatility. Secondly, Heston model will introduce stochastic volatility into the Black-Scholes' system. We will discuss some application of these models with respect to option pricing.
We will also discuss a simple model in ruin theory and systematic risk in banking system from stochastic approach.

\newpage

\section{Introduction}
% why simulation, benefit, ??
% choose matlab
The report will focus mainly on three topics 

\section{Black-Scholes' Economy}
The most basic model from the stochastic approach to model stock price is to use the Black-Schole-Merton Model. In this model, the stock price $(S_t)$ is given as
$$S_t = S_0 exp\left( (r - \frac{\sigma^2}{2} )t + \sigma W_t \right)$$
where $r$ is the interest rate, $\sigma$ is the volatility of that stock and $W$ is the standard Brownian motion. Therefore the logarithm of the stock price is a Brownian motion with drift, namely 
$$\log(S_t) = \log(S_0) \times ((r - \frac{\sigma^2}{2} )t + \sigma W_t )$$

As we will see later, computer simulation of the Brownian motion is crucial for some numerical methods to evaluate some financial assets, such as exotic options. 

There are several ways to simulate a standard Brownian motion. A straightforward approach is to use the first order Euler scheme and discretise the unit time to $N_{int}$ intervals. By the property of independent and stationary increments of Brownian motion, we can determine the distribution of the process at each point given its previous location. Mathematically, we denote $X$ as $\log(S_t)$ and
$$ \hat{X}_{(j+1)h} = \hat{X}_{jh} +( r-\frac{\sigma^2}{2})h +\sqrt{h} \sigma Z_{j+1}\  \forall j = 0, 1, \dots $$
where $h = \frac{1}{N_{int}}$ is the time interval between every sampling of the process and $Z_j$s are independent and identically distributed normal random variables with mean 0 and variance 1.

Given the same initial price and interest rate, the following figure(Figure. \ref{fig:1.stock_vs_vol})gives the histgram the the $\log(S_t)$ by several different volatilities. 
\begin{figure}[t]
        \centering
        \begin{subfigure}[t]{0.45\textwidth}
                \includegraphics[width=\textwidth]{./fig/{1.logST_hist_sigma0.2}.pdf}
                \caption{$\sigma = 0.2$}
        \end{subfigure}%
        ~ %add desired spacing between images, e. g. ~, \quad, \qquad etc.
          %(or a blank line to force the subfigure onto a new line)
        \begin{subfigure}[t]{0.45\textwidth}
                \includegraphics[width=\textwidth]{./fig/{1.logST_hist_sigma0.4}.pdf}
                \caption{$\sigma = 0.4$}
        \end{subfigure}
		~
        \begin{subfigure}[t]{0.45\textwidth}
                \includegraphics[width=\textwidth]{./fig/{1.logST_hist_sigma0.6}.pdf}
                \caption{$\sigma = 0.6$}
        \end{subfigure}
        \begin{subfigure}[t]{0.45\textwidth}
                \includegraphics[width=\textwidth]{./fig/{1.logST_hist_sigma0.8}.pdf}
                \caption{$\sigma = 0.8$}
        \end{subfigure}
        \caption{log stock price with different volatilities}
		\label{fig:1.stock_vs_vol}
\end{figure}

\section{Heston Model}

\section{Ruin Probability in Finite Time}

\section{Systematic Risk}


\section{Conclusion}




%\bibliographystyle{agsm}
%\bibliography{Reference}

\end{document}

