% -----------------------------------------
\documentclass[a4paper, 11pt]{article}

\usepackage[utf8x]{inputenc}
\usepackage{amsmath,amssymb,amsthm,amsfonts}
\usepackage[left=3.5cm, right=2cm]{geometry}
\usepackage{fullpage}
\usepackage{amsmath}
\usepackage[pdftex]{graphicx}%image support.
\usepackage{caption}
\usepackage{subcaption}
\usepackage{comment}
\usepackage{natbib}
\usepackage{fancyhdr}
\usepackage[hidelinks]{hyperref}
\usepackage{setspace}
\usepackage{upquote}
\usepackage{listings}
\usepackage{color}

\definecolor{dkgreen}{rgb}{0,0.6,0}
\definecolor{gray}{rgb}{0.5,0.5,0.5}
\definecolor{mauve}{rgb}{0.58,0,0.82}

\lstset{frame=tb,
	language=Matlab,
	aboveskip=3mm,
	belowskip=3mm,
	showstringspaces=false,
	columns=flexible,
	basicstyle={\small\ttfamily},
	numbers=none,
	numberstyle=\tiny\color{gray},
	keywordstyle=\color{blue},
	commentstyle=\color{dkgreen},
	stringstyle=\color{mauve},
	breaklines=true,
	breakatwhitespace=true
	tabsize=2
}
\newcommand{\includecode}[1]{\lstinputlisting[caption=#1, escapechar=,]{./code/#1}}

\onehalfspacing

\setlength{\headsep}{15pt}
\setlength{\headheight}{15pt}
\pagestyle{fancy}
\lhead{}\chead{}\rhead{17746}
\lfoot{}\cfoot{\thepage}\rfoot{}

\graphicspath{{./fig}}

\begin{document}
\title{Department of Statistics 2014\\ {\bf Project in Computational Finance and Insurance}\\ {\small Submitted for the Master of Science, London School of Economics} }
\author{Candidate Number: \bf 17746}
%\date{}
\maketitle
\vfill
\section*{Abstract}
This project will simlulate several interesting models used in financial markets. We will start with the famous Black-Scholes' model with constant volatility. Secondly, Heston model will introduce stochastic volatility into the Black-Scholes' system. We will discuss some application of these models with respect to option pricing.
We will also discuss a simple model in ruin theory and systematic risk in banking system from stochastic approach.

\newpage

\section{Introduction}
% why simulation, benefit, ??
% choose matlab
The report will focus mainly on three topics 

\section{Black-Scholes' Economy}
The most basic model from the stochastic approach to model stock price is to use the Black-Schole-Merton Model. In this model, the stock price $(S_t)$ is given as
$$S_t = S_0 \exp\left( (r - \frac{\sigma^2}{2} )t + \sigma W_t \right)$$
where $r$ is the interest rate, $\sigma$ is the volatility of that stock and $W$ is the standard Brownian motion. Therefore the logarithm of the stock price is a Brownian motion with drift, namely 
$$\log(S_t) = \log(S_0) + ((r - \frac{\sigma^2}{2} )t + \sigma W_t )$$

As we will see later, computer simulation of the Brownian motion is crucial for some numerical methods to evaluate some financial assets, such as exotic options. 

There are several ways to simulate a standard Brownian motion. A straightforward approach is to use the first order Euler scheme and discretise the unit time to $N_{int}$ intervals. By the property of independent and stationary increments of Brownian motion, we can determine the distribution of the process at each point given its previous location. Mathematically, we denote $X$ as $\log(S_t)$ and
$$ \hat{X}_{(j+1)h} = \hat{X}_{jh} +( r-\frac{\sigma^2}{2})h +\sqrt{h} \sigma Z_{j+1}\  \forall j = 0, 1, \dots $$
where $h = \frac{1}{N_{int}}$ is the time interval between every sampling of the process and $Z_j$s are independent and identically distributed normal random variables with mean 0 and variance 1.

Given the same initial price and interest rate, the following figure(Figure. \ref{fig:1.stock_vs_vol})gives the histgram the the $\frac{1}{\sigma}\log(S_t)$ by several different volatilities. As we can see from the figures, as the volatility increases, the mean started to decrease and the spread of the distribution remained similar. This is also clear from its formulation as 
$$\frac{1}{\sigma}\log(S_t) = \log(S_0) + (\frac{r}{\sigma} - \frac{\sigma}{2} )t + W_t$$
and increase in volatility negatively affect the linear drift term in above formula.

For some exotic options, direct pricing using risk neutral measure method could be difficult. Another way is to use Monte Carlo simulation. A simple implementation of this method is to price a European call option. 

\begin{figure}[t]
        \centering
        \begin{subfigure}[t]{0.45\textwidth}
                \includegraphics[width=\textwidth]{./fig/{1.logST_hist_sigma0.2}.pdf}
                \caption{$\sigma = 0.2$}
        \end{subfigure}%
        ~ %add desired spacing between images, e. g. ~, \quad, \qquad etc.
          %(or a blank line to force the subfigure onto a new line)
        \begin{subfigure}[t]{0.45\textwidth}
                \includegraphics[width=\textwidth]{./fig/{1.logST_hist_sigma0.4}.pdf}
                \caption{$\sigma = 0.4$}
        \end{subfigure}
		~
        \begin{subfigure}[t]{0.45\textwidth}
                \includegraphics[width=\textwidth]{./fig/{1.logST_hist_sigma0.6}.pdf}
                \caption{$\sigma = 0.6$}
        \end{subfigure}
        \begin{subfigure}[t]{0.45\textwidth}
                \includegraphics[width=\textwidth]{./fig/{1.logST_hist_sigma0.8}.pdf}
                \caption{$\sigma = 0.8$}
        \end{subfigure}
        \caption{log stock price with different volatilities}
		\label{fig:1.stock_vs_vol}
\end{figure}

\section{Heston Model}

\section{Ruin Probability in Finite Time}

\section{Systematic Risk}


\section{Conclusion}




%\bibliographystyle{agsm}
%\bibliography{Reference}

\section{Appendix}
Matlab codes used in the project:
\includecode{Q1.m}
\includecode{GeometricBrownianMotion.m}
\includecode{CallValue.m}
\includecode{CallPricingByMonteCarlo.m}
\includecode{Q2.m}
\includecode{getHestonSimulation.m}
\includecode{HestonCallPricingByMonteCarlo.m}
\includecode{getImpliedVolatility.m}
\includecode{DownAndOutOptionValue.m}
\includecode{DownAndOutOptionPricingByMonteCarlo.m}
\includecode{DownAndOutOptionPricingBySolvingPDE.m}
\includecode{Q3.m}
\includecode{CompoundPoissonGenerator.m}
\includecode{getU.m}
\includecode{ruinProbSimulation.m}
\includecode{getTau.m}
\includecode{getNewU.m}
\includecode{ComputeETTau.m}
\includecode{isRuined.m}
\includecode{notRuined.m}
\includecode{SystemRiskModel.m}
\includecode{getDefaultTime.m}
\includecode{DefaultTimeDiffSummary.m}
\includecode{getNumofDefault.m}
\includecode{saveTightFigure.m}
\includecode{LSummary.m}
\end{document}

